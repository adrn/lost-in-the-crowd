% NOTES:
% -

\documentclass[twocolumn]{AASTeX7/aastex701}

\shorttitle{TODO: short title}
\shortauthors{Price-Whelan et al.}

\usepackage{preamble/preamble-adrn}

\begin{document}

\title{
  The evolution of disrupted wide binary stars: \\
  The impact of dynamical noise
}

\author[orcid=0000-0003-0872-7098]{Adrian M. Price-Whelan}
\email{aprice-whelan@flatironinstitute.org}
\affiliation{
  Center for Computational Astrophysics, Flatiron Institute,
  162 Fifth Ave, New York, NY 10010, USA
}

\begin{abstract}
Wide binary star systems --- ``wide binaries'' --- in the Milky Way are gradually
disrupted by a combination of the smooth tidal field and stochastic perturbations from
passing stars and other perturbers (i.e., gravitational noise).
While the survival statistics and dynamics of bound wide binaries have been extensively
studied, the subsequent evolution of disrupted pairs --- ``comoving stars'' --- has not
been studied in realistic Galactic environments.
Here we develop an analytical framework for predicting the post-disruption phase-space
evolution of ionized wide binaries under the combined influence of the Galactic tidal
field and stochastic velocity perturbations.
We model the relative motion as a six-dimensional stochastic process and derive
closed-form solutions for the time-dependent covariance matrix and marginal
distributions of observable quantities.
Our solutions predict BLAH.
We validate the analytical results using numerical integration of the stochastic
differential equations and discuss the domain of validity of our linear approximation.
This framework provides a foundation for interpreting the growing catalog of comoving
stellar pairs from \textit{Gaia} as sensitive tracers of Galactic structure and the
spectrum of gravitational perturbations from the interstellar medium, dark matter
subhalos, and other sources.
\end{abstract}

% =====================================================================================

%--------------------------------------------------------------------
\section{Introduction}
\label{sec:intro}

\todo{Write this.}

%--------------------------------------------------------------------
\section{Data}
\label{sec:data}

\todo{Write this.}

%--------------------------------------------------------------------
% ... more sections

%--------------------------------------------------------------------
\section{Discussion}
\label{sec:discussion}

\todo{Write this.}

%--------------------------------------------------------------------
\section{Summary and Conclusions}
\label{sec:summary}

\todo{Write this.}

%--------------------------------------------------------------------
% \appendix

% \section{TODO name}
% \label{apx:name}

% \todo{Write this.}

% ====================================================================================

\begin{acknowledgments}
\todo{Write this.}
\end{acknowledgments}

% \begin{contribution}

%%SC was responsible for writing and submitting the manuscript.
%%WWM came up with the initial research concept and edited the manuscript.
%%OTS obtained the funding and edited the manuscript.
%%EBF provided the formal analysis and validation. He also edited the manuscript.
%%GEH Supervised the undergraduates, wrote the software and administers the project github and Zenodo repositories.
%%
%% Authors can use the Contributor Role Taxonomy (CRediT) at
%% https://credit.niso.org
%% for ideas on how write a good statement tailored to their needs.

% \end{contribution}

% \facilities{HST(STIS), Swift(XRT and UVOT), AAVSO, CTIO:1.3m, CTIO:1.5m, CXO}

% \software{astropy \citep{2013A&A...558A..33A,2018AJ....156..123A,2022ApJ...935..167A},
%           Cloudy \citep{2013RMxAA..49..137F},
%           Source Extractor \citep{1996A&AS..117..393B}
%           }

% \appendix
% \section{Appendix section example}


\bibliography{refs}{}
\bibliographystyle{AASTeX7/aasjournalv7}


\end{document}

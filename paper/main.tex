% NOTES:
% -

\documentclass[twocolumn]{AASTeX7/aastex701}

\shorttitle{TODO: short title}
\shortauthors{Price-Whelan et al.}

\usepackage{preamble/preamble-adrn}

\begin{document}

\title{
  Lost in the Crowd: \\
  The impact of dynamical noise on the evolution of disrupted wide binary stars
}

\author[orcid=0000-0003-0872-7098]{Adrian M. Price-Whelan}
\email{aprice-whelan@flatironinstitute.org}
\affiliation{
  Center for Computational Astrophysics, Flatiron Institute,
  162 Fifth Ave, New York, NY 10010, USA
}

\begin{abstract}
Wide binary star systems --- ``wide binaries'' --- in the Milky Way are gradually
disrupted by a combination of the smooth tidal field and stochastic perturbations from
passing stars and other perturbers (i.e., gravitational noise).
While the survival statistics and dynamics of bound wide binaries have been extensively
studied, the subsequent evolution of disrupted pairs --- ``comoving stars'' --- has not
been studied in realistic Galactic environments.
Here we develop an analytical framework for predicting the post-disruption phase-space
evolution of ionized wide binaries under the combined influence of the Galactic tidal
field and stochastic velocity perturbations.
We model the relative motion as a six-dimensional stochastic process and derive
closed-form solutions for the time-dependent covariance matrix and marginal
distributions of observable quantities.
Our solutions predict \todo{BLAH}.
We validate the analytical results using numerical integration of the stochastic
differential equations and discuss the domain of validity of our linear approximation.
This framework provides a foundation for interpreting the growing catalog of comoving
stellar pairs from \textit{Gaia} as sensitive tracers of Galactic structure and the
spectrum of gravitational perturbations from the interstellar medium, dark matter
subhalos, and other sources.
\end{abstract}

% =====================================================================================

%--------------------------------------------------------------------
\section{Introduction}
\label{sec:intro}

\todo{Write this.}

%--------------------------------------------------------------------
\section{Setup and Methods}
\label{sec:setup}

We will consider a disrupted binary-star system consisting of two stars with masses
$m_1$ and $m_2$ at positions $\vec{x}_1$ and $\vec{x}_2$ in the Galaxy.
The center of mass position (i.e. the barycenter) of the pair is then given by
\begin{equation}
  \vec{x}_{\mathrm{com}} =
    \frac{m_1 \, \vec{x}_1 + m_2 \, \vec{x}_2}{m_1 + m_2} \quad .
\end{equation}
We will assume that the stars orbit within the (assumed smooth and static) Galactic
gravitational potential $\Phi(\vec{x})$ with Galactocentric velocities $\vec{v}_1$ and
$\vec{v}_2$ and no mutual gravitational interaction (i.e., they act like two independent
test particles in the potential).
We define the relative position and velocity of the two stars as
\begin{align}
  \vec{\delta} &= \vec{x}_1 - \vec{x}_2 \\
  \vec{\gamma} &= \vec{v}_1 - \vec{v}_2 \quad .
\end{align}
The equation of motion of the relative separation is then
\begin{align}
  \ddot{\vec{\delta}} &=
    - \nabla \Phi(\vec{x}_1)
    + \nabla \Phi(\vec{x}_2)
    + \vec{F}(t)  \label{eq:eom1} \\
  &= \vec{F}_{\mathrm{tidal}} + \vec{F}(t) \quad ,
\end{align}
where $\vec{F}(t)$ represents the acceleration from anything not captured by the smooth,
static potential $\Phi$.
For example, $\vec{F}(t)$ could represent stochastic perturbations from passing stars,
molecular clouds, dark matter substructure, or periodic forcing from the Galactic bar,
or global secular changes to the potential.

When the relative separation $\vec{\delta}$ is small compared to the scale over which
the Galactic potential varies, we can Taylor expand the potential difference that
appears in Equation~\ref{eq:eom1} and keep only the leading-order term,
\begin{equation}
  \vec{F}_{\mathrm{tidal}} =
    - \nabla \Phi(\vec{x}_1)
    + \nabla \Phi(\vec{x}_2)
    \approx -\mat{T} \cdot \vec{\delta} \quad ,
\end{equation}
where the tidal tensor $\mat{T}$ is defined as
\begin{equation}
  T_{ij} \equiv
    \frac{\partial^2 \Phi}{\partial x_i \partial x_j}
    \bigg|_{\vec{x} = \vec{x}_{\mathrm{com}}} \quad .
\end{equation}
The tidal tensor is symmetric and its eigenvalues and eigenvectors describe the
principal directions of tidal compression and expansion at the location of the binary's
barycenter \citep[see, e.g.,][]{Binney:2008}.
The trace of the tidal tensor is related to the local mass density via Poisson's
equation,
\begin{equation}
  \mathrm{Tr}(\mat{T}) = \nabla^2 \Phi =
    4 \pi \, G \, \rho(\vec{x}_{\mathrm{com}}) \quad ,
\end{equation}
so in regions of low density (e.g., far from the Galactic plane) the tidal field is
approximately traceless.

This linearized approximation of the equations of motion is valid as long as the
separation $|\vec{\delta}|$ is small compared to the relevant scales of $\Phi$.
For a pair of stars orbiting in the Galactic disk, the smallest relevant scale is the
vertical scale height of the disk, which is around a few hundred parsecs \citep{TODO}.
In the stellar halo, the relevant scale is the scale radius of the dark matter halo,
which is significantly larger (of order 10 kpc; \citealt{TODO}).
Thus, for pairs with separations up to around a parsec to a few tens of parsecs, this
linearized approximation should be accurate.
\todo{Do we want to do a better job quantifying the induced error?}

The eigenvalues and eigenvectors of the tidal tensor determine the character of motion
of the relative separation, as is standard in orbital stability analysis.
Positive eigenvalues correspond to a restoring force, which leads to oscillatory motion
along the corresponding eigenvector direction, while negative eigenvalues correspond to
a repulsion that leads to exponential growth of the relative separation along the
corresponding direction.
However, in general, the tidal tensor is time-dependent in that it is evaluated at the
barycenter position of a pair of stars under consideration, and the barycenter orbits
and changes position within the Galaxy.

\todo{Two relevant approximations: (1) static tidal tensor in an inertial frame (valid for short times after disruption), (2) shearing sheet / epicyclic approximation (valid for longer times, but assumes circular orbit of barycenter).}

A common approximation for studying the dynamics of pairs or ensembles of stars in the
Galaxy is to work in the shearing sheet or epicyclic approximation
\citep[e.g.,][]{TODO, Oort}.
In this approximation, one works in a local Cartesian frame that co-rotates with the
Galaxy at a fixed radius, assuming that the barycenter of the pair of stars is on
a circular orbit.
The tidal tensor is then constant and diagonal, but the equations of motion include
additional Coriolis and centrifugal terms because of the rotating frame.




%--------------------------------------------------------------------
% ... more sections

%--------------------------------------------------------------------
\section{Discussion}
\label{sec:discussion}


\subsection{Relation to linear response theory} \label{sec:linresp}

The framework we have developed is an application of linear response theory to the
internal dynamics of wide binaries.
The transfer function $H(\omega)$ plays the role of a ``susceptibility,'' with poles at
the natural frequencies of the system.
Here, the natural frequencies of the system (i.e., the disrupted binary) are simply the
eigenvalues of the tidal tensor, which determine the characteristic frequencies of
oscillation or exponential growth of the relative separation.
In this way, this formalism is analogous to the response theory developed for stellar
systems \citep[e.g.,][]{Weinberg:1989, Hamilton?, Tremaine?}.
Perturbations from the noise couple most strongly to orbits that resonate with the
forcing frequency.
\todo{more to say?}

%--------------------------------------------------------------------
\section{Summary and Conclusions}
\label{sec:summary}

\todo{Write this.}

%--------------------------------------------------------------------
% \appendix

% \section{TODO name}
% \label{apx:name}

% \todo{Write this.}

% ====================================================================================

\begin{acknowledgments}
\todo{Write this.}

Ryan Rubenzahl

\end{acknowledgments}

% \begin{contribution}

%%SC was responsible for writing and submitting the manuscript.
%%WWM came up with the initial research concept and edited the manuscript.
%%OTS obtained the funding and edited the manuscript.
%%EBF provided the formal analysis and validation. He also edited the manuscript.
%%GEH Supervised the undergraduates, wrote the software and administers the project github and Zenodo repositories.
%%
%% Authors can use the Contributor Role Taxonomy (CRediT) at
%% https://credit.niso.org
%% for ideas on how write a good statement tailored to their needs.

% \end{contribution}

% \facilities{HST(STIS), Swift(XRT and UVOT), AAVSO, CTIO:1.3m, CTIO:1.5m, CXO}

% \software{astropy \citep{2013A&A...558A..33A,2018AJ....156..123A,2022ApJ...935..167A},
%           Cloudy \citep{2013RMxAA..49..137F},
%           Source Extractor \citep{1996A&AS..117..393B}
%           }

% \appendix
% \section{Appendix section example}


\bibliography{refs}{}
\bibliographystyle{AASTeX7/aasjournalv7}


\end{document}
